\documentclass[12]{article}
\usepackage{t1enc} % mindig az első sor
%láthatatlan beállítások rész PREAMBULUM
%filebeállítások, csomagtöltések
\usepackage[magyar]% kommentben sortörés sem dolgozza fel
%alapvetően panarncs nem tördelhet szőközzel, sortöréssel,de kommentel be lehet csapni
{babel}
\usepackage{amsmath}
\usepackage{xcolor}
\usepackage{graphicx}
\usepackage{hulipsum}
\usepackage{xcolor}
\usepackage{geometry}
\usepackage{fancyhdr}
\usepackage{enumitem}
\usepackage{hyperref}
\usepackage{enumitem}
\usepackage{lipsum}
%undefined control sequence: nem ismert parancs
%megoldás: elgépelés, vagy csomaghiány
%\usepackage{•}
% % = Komment
\pagestyle{fancy}
\fancyhf{} 
\geometry{left=1in, right=1in, top=1in, bottom=1in, headheight=15pt}

% Fejléc és lábléc beállítása
\pagestyle{fancy}
\fancyhf{} % Töröljük az alapértelmezett fejlécet és láblécet

% Fejléc beállítása páros oldalakon
\fancyhead[LE]{\leftmark} % Páros oldal: aktuális szakasz neve
\fancyhead[RE]{\rightmark} % Páros oldal: aktuális alfejezet neve
\fancyhead[RO]{\thepage} % Páros oldal: oldalszám a külső sarkában

% Fejléc beállítása páratlan oldalakon
\fancyhead[LO]{\thepage} 
\fancyhead[RE]{\leftmark} 

% Lábléc beállítása mindkét oldalon
\fancyfoot[C]{Miskolci egyetem}

% Választóvonal a lábléc fölé
\renewcommand{\footrulewidth}{0.4pt}
\renewcommand{\headrulewidth}{0pt} 

% Testreszabás a címoldalon
\fancypagestyle{plain}{%
  \fancyhf{} 
  \fancyfoot[RO]{\thepage} 
  \renewcommand{\footrulewidth}{0.4pt}
}

\begin{document}

%%címhez: KÜLÖN ADATBEVITEL + KIIRATÁS
%adatbevitel:
\title{Dokumentum címe} % lehet benne formázás, sortörés, stb.: pl.: alcím
\author{Takács Antal Levente}
\date{2024/2025\\I. félév}
% Üresen nincsen dátum

%kiíratás
\maketitle
\newpage

\section{1. Óra}
\subsection{1. Feladat}

$\emptyset$ % $ - "matematikai mód"

vezérlő karakterek: egyetlen speciális karakterből álló parancsok:
% $ " _ \ # &
backslash + spec karakterek
\$ \& ..... \'a \"a \H \\
Nem üres a dokumentumtörzs
		valami
	valami más

\textit{Dőlt betű} már nem% ctrl + i

{\itshape{dőlt betű} dőlt-e? yessir!
\\
ilyenkor blokk végéig, visszavonásig dőlt marad
\scshape kiskapitálisra váltottunk, blokk vége egyik sem érvényes már
}\par
\textit{dőlt betűs dőlt betű...
\textbf{vastag betű} }
%Olyat nem lehet, hogy \begin(A) \ begin(B) \end(A) \ end(B)
%Csak beágyazás, átlapolás nem !!!!!!
Új bekezdés.\par 
megint új bekezdés\\
sortörés\newline
másik sortörés

Egy mondat. Még egy mondat: és folytatjuk

\frenchspacing
Egy mondat. Még egy mondat: és folytatjuk

``magyar idézőjel" %bal: altgr + 7*2 ; jobb: altgr + szimpla aposztróf*2
\today % mai dátum

\selectlanguage{english}
``angol idézőjel" \ today
\newpage



\section{Feladat}
\subsection{valami}
\textbf{Ez a szöveg vastag betűs}\\
\textsc {Ez a szöveg kiskapitális}\\
\textit{Ez a szöveg dőlt betűs}\\
\textsl{Ez a szöveg salanted}\\
\textsf{Ez a szöveg sans serif}\\
\emph{\textbf{Ez a szöveg vastag betűs}}\\
\emph{\scshape {Ez a szöveg kiskapitális}}\\
\emph{\textit{Ez a szöveg dőlt betűs}}\\
\newpage



\section{Feladat}
\subsection{valami}
\textsf{egy mondaton belül emeljünk egy szót {\large nagyobb betűmérettel}, egy másik
szót pedig váltsunk \textsc {Csupa nagybetűre!}}\\
\newpage

\section{Feladat}
\subsection{valami}
\textcolor{pink}{pink szöveg}\\
\colorbox{pink}{szöveg pink háttérrel}\\
\reflectbox{\fcolorbox{blue}{white}{Függőleges tükrözés}}\\
\scalebox{1}[-1]{\scalebox{2.0}{tükrözött szöveg.}}\\
\rotatebox{90}{90°-al elforgatott szöveg}\\
\rotatebox{270}{270°-al elforgatott szöveg}\\
\textcolor{white}{\colorbox{black}{Invertált szöveg.}}\\
\fcolorbox{red}{white}{\textcolor{red}{piros szöveg, piros kerettel}}\\
\newpage

\section{feladat}
\subsection{2.Óra}

\setlist[itemize,1]{label=$\clubsuit$}

\begin{itemize}
\item Első elem
\item Második elem 
\item Harmadik elem
\end{itemize}
\newpage

\section{feladat}
\subsection{2.Óra}


\begin{enumerate}
    \item Első szint
    \begin{enumerate}
        \item Második szint
        \begin{enumerate}[label=\alph*),ref=\alph*]
            \item Harmadik szint
            \begin{enumerate}
                \item Negyedik szint
                \end{enumerate}
            \end{enumerate}
        \end{enumerate}
    \end{enumerate}
   
\newlist{myenum}{enumerate}{5}  % 5 szint engedélyezése
\setlist[myenum]{label=(\arabic*), resume}


\section{Saját myenum lista}

\begin{myenum}[resume]
    \item Első szint
    \begin{myenum}
        \item Második szint
        \begin{myenum}
            \item Harmadik szint
            \begin{myenum}
                \item Negyedik szint
                \begin{myenum}
                    \item Ötödik szint (extra szint)
                \end{myenum}
            \end{myenum}
        \end{myenum}
    \end{myenum}
\end{myenum}


Lorem Ipsum is simply dummy text of the printing and typesetting industry. Lorem Ipsum has been the industry's standard dummy text ever since the 1500s, when an unknown printer took a galley of type and scrambled it to make a type specimen book. It has survived not only five centuries, but also the leap into electronic typesetting, remaining essentially unchanged. It was popularised in the 1960s with the release of Letraset sheets containing Lorem Ipsum passages, and more recently with desktop publishing software like Aldus PageMaker including versions of Lorem Ipsum.

\begin{myenum}[resume]
    \item Első szint
    \begin{myenum}
        \item Második szint
        \begin{myenum}
            \item Harmadik szint
            \begin{myenum}
                \item Negyedik szint
                \begin{myenum}
                    \item Ötödik szint (extra szint)
                \end{myenum}
            \end{myenum}
        \end{myenum}
    \end{myenum}
\end{myenum}
\newpage

\section{Leíró lista}

\begin{description}[style=nextline]  % Leíró lista a "nextline" stílusban
    \item[\slshape ] \lipsum[1]  % Üres címke dőlt szöveggel
    \item[\slshape Rövid] \lipsum[2]  % Rövid címke dőlt szöveggel
    \item[\slshape Ez egy nagyon hosszú címke, ami több sort is igénybe vehet] \lipsum[3]  % Hosszú címke dőlt szöveggel
\end{description}

\end{document}