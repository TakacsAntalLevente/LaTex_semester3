\documentclass[12]{article}
\usepackage{t1enc} % mindig az első sor
%láthatatlan beállítások rész PREAMBULUM
%filebeállítások, csomagtöltések
\usepackage[magyar]% kommentben sortörés sem dolgozza fel
%alapvetően panarncs nem tördelhet szőközzel, sortöréssel,de kommentel be lehet csapni
{babel}
\usepackage{amsmath}
\usepackage{xcolor}
\usepackage{colortbl}
\usepackage{graphicx}
\usepackage{hulipsum}
\usepackage{xcolor}
\usepackage{geometry}
\usepackage{fancyhdr}
\usepackage{enumitem}
\usepackage{hyperref}
\usepackage{enumitem}
\usepackage{lipsum}
\usepackage{subcaption}
\usepackage{array}
\usepackage{multirow}
\usepackage{wrapfig}
\usepackage{float}
\usepackage{listings}
\usepackage{booktabs}
\usepackage{verbatim}
\usepackage[utf8]{inputenc}
%undefined control sequence: nem ismert parancs
%megoldás: elgépelés, vagy csomaghiány
%\usepackage{•}
% % = Komment
\pagestyle{fancy}
\fancyhf{} 
\geometry{left=1in, right=1in, top=1in, bottom=1in, headheight=15pt}

% Fejléc és lábléc beállítása
\pagestyle{fancy}
\fancyhf{} % Töröljük az alapértelmezett fejlécet és láblécet

% Fejléc beállítása páros oldalakon
\fancyhead[LE]{\leftmark} % Páros oldal: aktuális szakasz neve
\fancyhead[RE]{\rightmark} % Páros oldal: aktuális alfejezet neve
\fancyhead[RO]{\thepage} % Páros oldal: oldalszám a külső sarkában

% Fejléc beállítása páratlan oldalakon
\fancyhead[LO]{\thepage} 
\fancyhead[RE]{\leftmark} 

% Lábléc beállítása mindkét oldalon
\fancyfoot[C]{Miskolci egyetem}

% Választóvonal a lábléc fölé
\renewcommand{\footrulewidth}{0.4pt}
\renewcommand{\headrulewidth}{0pt} 

% Testreszabás a címoldalon
\fancypagestyle{plain}{%
  \fancyhf{} 
  \fancyfoot[RO]{\thepage} 
  \renewcommand{\footrulewidth}{0.4pt}
}

\begin{document}

%Képjegyzék
\listoffigures
\contentsline {figure}{\numberline {1}{\ignorespaces Példa kép}}{9}{figure.caption.2}
\contentsline {figure}{\numberline {2}{\ignorespaces Normális kép}}{11}{figure.caption.4}
\contentsline {figure}{\numberline {3}{\ignorespaces Valami}}{11}{figure.caption.4}

%Táblázat jegyzék
\listoftables

%%címhez: KÜLÖN ADATBEVITEL + KIIRATÁS
%adatbevitel:
\title{Dokumentum címe} % lehet benne formázás, sortörés, stb.: pl.: alcím
\author{Takács Antal Levente}
\date{2024/2025\\I. félév}
% Üresen nincsen dátum

%kiíratás
\maketitle





\newpage

\section{3.óra 1.feladat}

\begin{figure}[h]
    \centering
    \includegraphics[width=5cm, height=5cm, keepaspectratio]{C:/Users/au090647/Desktop/Új mappa/3.óra/szines.jpg}
    \caption{Példa kép}
\end{figure}


\lipsum[1]


Itt látható a kép \includegraphics[width=5cm, keepaspectratio]{C:/Users/au090647/Desktop/Új mappa/3.óra/szines.jpg} és folytatódik a szöveg itt mellette. \lipsum[2]


\lipsum[3]
\newpage

\lipsum[1]


Itt látható a kép \includegraphics[width=5cm, keepaspectratio]{C:/Users/au090647/Desktop/Új mappa/3.óra/szines.jpg} és folytatódik a szöveg itt mellette. \lipsum[2]


\lipsum[3]


\begin{figure}[htbp]
    \centering
Itt látható a kép \includegraphics[width=5cm, keepaspectratio]{C:/Users/au090647/Desktop/Új mappa/3.óra/szepia.jpg}
\caption{Úszó kép}  és folytatódik a szöveg itt mellette. és folytatódik a szöveg itt mellette. \lipsum[4]
\end{figure}


\lipsum[4]
\newpage

\lipsum[1]


Itt látható a kép \includegraphics[width=5cm, keepaspectratio]{C:/Users/au090647/Desktop/Új mappa/3.óra/szines.jpg} és folytatódik a szöveg itt mellette. \lipsum[2]

% Második bekezdés lorem ipsum
\lipsum[3]


\begin{figure}[htbp]
    \centering
    \includegraphics[width=5cm, keepaspectratio]{C:/Users/au090647/Desktop/Új mappa/3.óra/szepia.jpg}
    \includegraphics[width=5cm, keepaspectratio]{C:/Users/au090647/Desktop/Új mappa/3.óra/szines.jpg}
    \caption{Normális kép}
    \reflectbox{\includegraphics[width=5cm, keepaspectratio]{C:/Users/au090647/Desktop/Új mappa/3.óra/szines.jpg}}
    \caption{Tükrözött kép}
\end{figure}


\lipsum[4]
\newpage

\begin{figure}[htbp]
    \centering
    \begin{subfigure}{0.45\textwidth}
        \centering
        \includegraphics[width=\linewidth, keepaspectratio]{C:/Users/au090647/Desktop/Új mappa/3.óra/szepia.jpg}
        \caption{Első részábra felirata}
        \label{fig:sub1} % Részábra címke
    \end{subfigure}
    \hspace{0.05\textwidth} % Térköz
    \begin{subfigure}{0.45\textwidth}
        \centering
        \includegraphics[width=\linewidth, keepaspectratio]{C:/Users/au090647/Desktop/Új mappa/3.óra/szines.jpg}
        \caption{Második részábra felirata}
        \label{fig:sub2} % Második részábra címke
    \end{subfigure}
    \caption{Ez a teljes ábra felirata}
    \label{fig:main} % Fő ábra címke
\end{figure}

% Hivatkozás
A fő ábra a \ref{fig:main} számú ábra, amely a következő részábrákat tartalmazza: \subref{fig:sub1} és \subref{fig:sub2}.
\newpage

\section{3.óra 2.feladat}

\begin{table}[htbp]
    \centering
    \caption{1. Táblázat címe}
    \begin{tabular}{>{\centering\arraybackslash}p{30pt} | l | c | r}
        \toprule
        1. oszlop & 2. oszlop & 3. oszlop & 4. oszlop \\
        \midrule
        A1 & B1 & C1 & D1 \\
        A2 & B2 & & D2 \\
        & B3 & C3 & D3 \\
        A4 & & C4 & D4 \\
        \bottomrule
    \end{tabular}
\end{table}


% 1. másolata
\begin{table}[htbp]
    \centering
    \caption{1. Táblázat (bal szélső oszlop nélkül)}
    \begin{tabular}{l | c | r}
        \toprule
        1. oszlop & 2. oszlop & 3. oszlop \\
        \midrule
        B1 & C1 & D1 \\
        B2 & & D2 \\
        B3 & C3 & D3 \\
        & C4 & D4 \\
        \bottomrule
    \end{tabular}
\end{table}

% 2a váltakozó színezése
\begin{table}[htbp]
    \centering
    \caption{2a. Táblázat: Váltakozó színezés}
    \begin{tabular}{|l|c|r|}
        \hline
        1. oszlop & 2. oszlop & 3. oszlop \\
        \hline
        \rowcolor{gray!20} B1 & C1 & D1 \\
        \hline
        \rowcolor{white} B2 & C2 & D2 \\
        \hline
        \rowcolor{gray!20} B3 & C3 & D3 \\
        \hline
        \rowcolor{white} B4 & C4 & D4 \\
        \hline
    \end{tabular}
\end{table}

% 2b.színezett cellák
\begin{table}[htbp]
    \centering
    \caption{2b. Táblázat: Színezett cellák}
    \begin{tabular}{|>{\columncolor{red!20}}l|>{\columncolor{blue!20}}c|>{\columncolor{green!20}}r|}
        \hline
        1. oszlop & 2. oszlop & 3. oszlop \\
        \hline
        B1 & C1 & D1 \\
        \hline
        B2 & C2 & D2 \\
        \hline
        B3 & C3 & D3 \\
        \hline
        B4 & C4 & D4 \\
        \hline
    \end{tabular}
\end{table}

\begin{table}[htbp]
    \centering
    \caption{3. Táblázat: Cellák egyesítése}
    \begin{tabular}{|c|c|c|}
        \hline
        \multicolumn{2}{|c|}{1,2 Egyesített oszlop} & 3. oszlop \\ \hline
        \multirow{2}{*}{Függőleges} & C1 & D1 \\ \cline{2-3}
        & C2 & 2. D2 \\ \hline
       B3 & C3 &D3 \\ \hline
        B4 & \multicolumn{2}{c|}{2,3Egyesített oszlop} \\ \hline
    \end{tabular}
\end{table}


\begin{wraptable}{r}{0.5\textwidth} % 'r' a jobb, 'l' a bal oldalra helyezéshez
    \centering
    \caption{3. Táblázat: Cellák egyesítése}
    \begin{tabular}{|c|c|c|}
        \hline
        \multicolumn{2}{|c|}{1,2 Egyesített oszlop} & 3. oszlop \\ \hline
        \multirow{2}{*}{Függőleges} & C1 & D1 \\ \cline{2-3}
        & C2 & 2. D2 \\ \hline
       B3 & C3 &D3 \\ \hline
        B4 & \multicolumn{2}{c|}{2,3Egyesített oszlop} \\ \hline
    \end{tabular}
\end{wraptable}

\lipsum[5]
\newpage

\section{3.óra 3.feladat}

\begin{verbatim}
\begin{enumerate}
    \item Első elem
    \item Második elem
    \item Harmadik elem
\end{enumerate}
\end{verbatim}
\newpage

\section{3.óra 4.feladat(Programkód 1)}


\lstset{ 
    language=Python,                 % kód nyelv
    backgroundcolor=\color{white},   % Háttér
    basicstyle=\footnotesize,        % betűméret
    breaklines=true,                 % Sortörés engedélyezése
    frame=single,                    % Keret
    keywordstyle=\color{blue},       % Kulcsszavak színe
    commentstyle=\color{green},      % Megjegyzések színe
    stringstyle=\color{red},         % Stringek színe
    tabsize=2,                       % Tabulátorszélesség 2 szóköz
    columns=flexible,                % Rugalmas oszlopok
    showstringspaces=false,          % Szóközök
    numbers=left,                    % Sorszámok
    numberstyle=\tiny\color{gray},   % Számozás stílus
    numbersep=5pt,                   % Távolság a számoktól
    stepnumber=4,                    % sorszám 4
    frame=lines                   	 % Keret
}

\begin{lstlisting}[language=Python]
def binary_search(arr, val, start, end):
    if start == end:
        if arr[start] > val:
            return start
        else:
            return start + 1
    elif start > end:
        return start
    else: 
        mid = (start + end) // 2   Hasznalj egesz osztast
        if arr[mid] < val:
            return binary_search(arr, val, mid + 1, end)
        elif arr[mid] > val:
            return binary_search(arr, val, start, mid - 1)
        else:  # arr[mid] = val
            return mid
            
def insertion_sort(arr):
    for i in range(1, len(arr)):   Hasznalj 'range' helyett 'xrange'
        val = arr[i]
        j = binary_search(arr, val, 0, i - 1)
        arr = arr[:j] + [val] + arr[j:i] + arr[i + 1:]
    return arr
\end{lstlisting}
\newpage

\section{3.óra 5.feladat(Programkód 2)}

\lstset{ 
    language=Python,                 		% kód nyelv
     backgroundcolor=\color{lightgray},   	% Háttér
    basicstyle=\footnotesize,        		% betűméret
    breaklines=true,                 		% Sortörés engedélyezése
    frame=single,                   		% Keret
    keywordstyle=\color{red},       		% Kulcsszavak színe
    commentstyle=\color{gray},      		% Megjegyzések színe
    stringstyle=\color{red},         		% Stringek színe
    tabsize=2,                       		% Tabulátorszélesség 2 szóköz
    columns=flexible,                		% Rugalmas oszlopok
    showstringspaces=false,          		% Szóközök
    numbers=left,                    		% Sorszámok
    numberstyle=\tiny\color{gray},   		% Számozás stílus
    numbersep=5pt,                   		% Távolság a számoktól
    stepnumber=4,                    		% sorszám 4
    frame=lines                   	 		% Keret
    morekeywords={def},              		% Függvények kiemelése
    keywordstyle=[2]\color{blue}, 			% Külön szín a 'def' kulcsszónak
}
    
\begin{lstlisting}

// this is just a code snippet saved from the internet
// for LaTeX code input testing

binarySearch(arr, x, low, high)
	repeat till low = high
		mid = (low + high)/2
			if (x == arr[mid])
				return mid

			else if (x > arr[mid])	// x is on the right side
				low = mid + 1

			else			// x is on the left side
				high = mid - 1

recursiveBinarySearch(arr, x, low, high)
	if low > high
		return False

	else
		mid = (low + high) / 2
		if x == arr[mid]
			return mid

		else if x > arr[mid]	// x is on the right side
			return binarySearch(arr, x, mid + 1, high)

		else			// x is on the left side
			return binarySearch(arr, x, low, mid - 1)

// ok bye
\end{lstlisting}
% Kód formázás megőrzése keret nélkül
\lstset{frame=none} % Keret eltávolítás
\newpage

\section{3.óra 6.feladat}

% Külön környezet Python kódoknak
\newfloat{pythoncode}{htbp}{lop}
\floatname{pythoncode}{Python Kód}

% Külön környezet C kódoknak
\newfloat{ccode}{htbp}{loc}
\floatname{ccode}{C Kód}

% Listings beállítások
\lstset{ 
    backgroundcolor=\color{lightgray}, % Háttérszín világos szürke
    basicstyle=\footnotesize,          % Alapértelmezett betűméret
    breaklines=true,                   % Sortörés engedélyezése
    tabsize=2,                         % Tabulátorszélesség 2 szóköz
    numbers=left,                      % Sorok számozása bal oldalon
    numberstyle=\tiny\color{gray},     % Számozás stílusa
}



\section*{Példa Kódok}

% Python kód környezet

\begin{pythoncode}
\caption{Példa Python kód}
\begin{lstlisting}[language=Python]

def hello_world():
    print("Hello, World!")
\end{lstlisting}
\end{pythoncode}

% C kód környezet
\begin{ccode}
\caption{Példa C kód}
\begin{lstlisting}[language=C]
#include <stdio.h>

int main() {
    printf("Hello, World!\n");
    return 0;
}
\end{lstlisting}
\end{ccode}

% Módosított C kód környezet
\begin{ccode}
\caption{Módosított C kód: Másik Függvény}
\begin{lstlisting}[language=C]
#include <stdio.h>

void hello_world() {
    printf("Hello, World!\n");
}

int main() {
    hello_world();
    return 0;
}
\end{lstlisting}
\end{ccode}
\end{document}