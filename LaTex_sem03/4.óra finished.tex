\documentclass{article}
\usepackage{amsthm} % tételhez szükséges csomag
\newtheorem{theorem}{Tétel}

\begin{document}

\begin{theorem}
Ez egy egyszerű tétel.
\end{theorem}

\begin{theorem}[Takács Antal Levente] 
Ez egy másik tétel szerzővel és névvel.
\end{theorem}

\newtheorem{lemma}[theorem]{Lemma} 

\begin{lemma}
Ez egy lemma.
\label{lem:lemma}
\end{lemma}

\theoremstyle{definition} % tétel stílusának megváltoztatása
\newtheorem{definition}{Definíció}[section] 

\section{Első szakasz}

\begin{definition}
Első definíció az első szakaszban.
\end{definition}

\begin{definition}
Második definíció az első szakaszban.
\end{definition}

\section{Második szakasz}

\begin{definition}
Első definíció a második szakaszban.
\end{definition}

\begin{proof}
Ez a bizonyítás a Lemma bizonyításához tartozik. Lásd \ref{lem:lemma}. % kereszthivatkozás
\end{proof}
\newpage

Az \( \frac{1}{n^2} \) sorösszege: 
\[
\sum_{n=1}^{\infty} \frac{1}{n^2} = \frac{\pi^2}{6}.
\]

A \( n! \) (n faktoriális) a számok szorzata 1-től n-ig, azaz:
\[
n! = \prod_{k=1}^{n} k = 1 \cdot 2 \cdot \ldots \cdot n.
\]

Legyen \( 0 \leq k \leq n \). A binomiális együttható:
\[
\binom{n}{k} = \frac{n!}{k!(n-k)!}.
\]

DeclareMathOperator*{\argmax}{arg\,max} % argmax operátor definiálása

x^* = \argmax_{x \in [0,1]} x \log_2(x)

\newcommand{\ceil}[1]{\left\lceil #1 \right\rceil}

\[
\ceil{x}, \quad \ceil{\frac{5}{3}}
\]

\newcommand{\E}[1]{\mathbb{E}\left[#1\right]}

\[
\E{X_i}, \quad \E{\sum_{i=1}^{N} X_i}
\]

\newcommand{\CE}[2]{\mathbb{E}\left[#1 \, \middle| \, #2\right]}

\[
\CE{X_i}{X_j}, \quad \CE{\sum_{i=1}^{N} X_i}{N}
\]
\newpage

\begin{align}
(a + b)^{n+1} &= (a+b) \cdot \left( \sum_{k=0}^n \binom{n}{k} a^{n-k}b^k \right) \\
              &= \sum_{k=0}^n \binom{n}{k} a^{n+1-k}b^k + \sum_{k=1}^{n+1} \binom{n}{k-1} a^{n+1-k}b^k
\end{align}


\begin{table}[htbp]
    \centering
    \caption{Táblázat példa}
    \begin{tabular}{|c|c|c|}
        \hline
        \multicolumn{2}{|c|}{1,2 Egyesített oszlop} & 3. oszlop \\ \hline
        \multirow{2}{*}{Függőleges} & C1 & D1 \\ \cline{2-3}
        & C2 & D2 \\ \hline
    \end{tabular}
\end{table}



\end{document}
