\documentclass{article}
\usepackage{t1enc} % mindig az első sor
%láthatatlan beállítások rész PREAMBULUM
%filebeállítások, csomagtöltések
\usepackage[magyar]% kommentben sortörés sem dolgozza fel
%alapvetően panarncs nem tördelhet szőközzel, sortöréssel,de kommentel be lehet csapni
{babel}
\usepackage{amsmath}
\usepackage{xcolor}
\usepackage{graphicx}
%undefined control sequence: nem ismert parancs
%megoldás: elgépelés, vagy csomaghiány
%\usepackage{•}
% % = Komment
\begin{document}

%%címhez: KÜLÖN ADATBEVITEL + KIIRATÁS
%adatbevitel:
\title{Dokumentum címe} % lehet benne formázás, sortörés, stb.: pl.: alcím
\author{Takács Antal Levente}
\date{2024/2025\\I. félév}
% Üresen nincsen dátum

%kiíratás
\maketitle 

$\emptyset$ % $ - "matematikai mód"

vezérlő karakterek: egyetlen speciális karakterből álló parancsok:
% $ " _ \ # &
backslash + spec karakterek
\$ \& ..... \'a \"a \H \\
Nem üres a dokumentumtörzs
		valami
	valami más

\textit{Dőlt betű} már nem% ctrl + i

{\itshape{dőlt betű} dőlt-e? yessir!
\\
ilyenkor blokk végéig, visszavonásig dőlt marad
\scshape kiskapitálisra váltottunk, blokk vége egyik sem érvényes már
}\par
\textit{dőlt betűs dőlt betű...
\textbf{vastag betű} }
%Olyat nem lehet, hogy \begin(A) \ begin(B) \end(A) \ end(B)
%Csak beágyazás, átlapolás nem !!!!!!
Új bekezdés.\par 
megint új bekezdés\\
sortörés\newline
másik sortörés

Egy mondat. Még egy mondat: és folytatjuk

\frenchspacing
Egy mondat. Még egy mondat: és folytatjuk

``magyar idézőjel" %bal: altgr + 7*2 ; jobb: altgr + szimpla aposztróf*2
\today % mai dátum

\selectlanguage{english}
``angol idézőjel" \ today

\begin{center}
{\Huge 2. Feladat}
\end{center}
\textbf{Ez a szöveg vastag betűs}\\
\textsc {Ez a szöveg kiskapitális}\\
\textit{Ez a szöveg dőlt betűs}\\
\textsl{Ez a szöveg salanted}\\
\textsf{Ez a szöveg sans serif}\\
\emph{\textbf{Ez a szöveg vastag betűs}}\\
\emph{\scshape {Ez a szöveg kiskapitális}}\\
\emph{\textit{Ez a szöveg dőlt betűs}}\\



\begin{center}
{\Huge 3. Feladat}
\end{center}
\textsf{egy mondaton belül emeljünk egy szót {\large nagyobb betűmérettel}, egy másik
szót pedig váltsunk \textsc {Csupa nagybetűre!}}\\

\begin{center}
{\Huge 4. Feladat}
\end{center}
\textcolor{pink}{pink szöveg}\\
\colorbox{pink}{szöveg pink háttérrel}\\
\reflectbox{\fcolorbox{blue}{white}{Függőleges tükrözés}}\\
\scalebox{1}[-1]{\scalebox{2.0}{tükrözött szöveg.}}\\
\rotatebox{90}{90°-al elforgatott szöveg}\\
\rotatebox{270}{270°-al elforgatott szöveg}\\
\textcolor{white}{\colorbox{black}{Invertált szöveg.}}\\
\fcolorbox{red}{white}{\textcolor{red}{piros szöveg, piros kerettel}}
\end{document}